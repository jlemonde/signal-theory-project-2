\documentclass[twocolumn, 12pt]{IEEEtran}

\usepackage[margin=0.5in]{geometry}

\usepackage{amsfonts,amsmath,amssymb,amsthm,bm,relsize}
\usepackage{balance}
\usepackage{graphicx}

\begin{document}

\title{Project Assignment 2: Report}
\author{Johannes Lemonde 960911-T357 and Lucas Streit 970606-T???}
mettre le numéro personnel de Lucas

\maketitle

\section{Introduction}

The main concern of this project is to decode a signal which was distorted by an unknown time-invariant finite impulse response filter and which includes also some white noise. Such filters are commonly used by communication channels; our role here is to take a glimpse into the maths behind the receptor -- in other words, to elaborate the algorithm able to decode this distorted signal back to the original signal.

We are provided with such a distorted signal consisting of a large sequence of real numbers, knowing that the original binary signal --~before it was distorted~-- starts with a given binary sequence, which we also know.

The knowledge of this sequence at the beginning of the signal is needed by the receptor to compute a set of constant parameters, comparing the original and distorted signals --~through a system of linear equations to solve~--, which would then be used to determine the further bits of the original signal.

To make this project fancier, we are dived into a situation where the original signal is a key to decipher a picture. We are given an encrypted picture and the ciphering/deciphering algorithms, and the better the recovered key, the more accurate the deciphered picture.

\section{Formalisation of the problem}

 The original binary signal -- the key -- is composed by a sequence ${b(k)\in \{-1,1\}}, ~~{k\in[1, M]\subset~\mathbb{N}}$, where $M$ is the length of the key.
 The values ${b(1),~\dots~, b(N)}$ --~referred to as the training sequence~-- are known to us, with ${N=32}$.
 Before it was transmitted, the key has been subjected to an unknown filter $h$ of order ${3}$ and to the white noise $n(k)$ such as 
 \begin{equation}\label{eq:r}
 { r(k) = \sum\limits_{l=0}^{3}h(l)~b(k-l)~+n(k),~~k\in[1,M]\subset\mathbb{N} }.
 \end{equation}
 These values of $r$ are known to us, and we are asked to reconstruct the sequence $b$.
 
 \section{Resolution}
 
 In order to do so, we apply an equaliser (an other filter) of order $L$ such as 
 \begin{equation}\label{eq:bt}
 { \hat{b}_{r}(k) = \sum\limits_{l=0}^{L}w(l)~r(k-l) \approx b(k) },
 \end{equation}
 where the coefficients ${w(0)},~{\dots~},{w(L+1)}$ must be initialised using the training sequence. To do so, we define a matrix $\bm{R}$, a vector $\bm{w}$ and a vector $\bm{b}$ so that $\bm{Rw}=\bm{b}$ represent the equations in (\ref{eq:bt}). Since $r(k)$ does not have values for $k$ being null or negative, and since we have this subtraction of $l$ in the argument of $r$ in (\ref{eq:bt}), we can only use (\ref{eq:bt}) for $k = L+1,~\dots~, N$, which thus gives $\bm{Rw}=\bm{b} \Leftrightarrow $
\begin{equation}\label{eq:matrixes}
\begin{pmatrix}
    \mathsmaller{r(L+1)}   & \mathsmaller{r(L)} & \mathsmaller{\cdots}  & \mathsmaller{r(1)} \\
    \mathsmaller{r(L+2)}   & \mathsmaller{r(L+1)} & \mathsmaller{\cdots}  & \mathsmaller{r(2)} \\
    \mathsmaller{\vdots}   & \mathsmaller{\vdots} & \mathsmaller{\cdots} & \mathsmaller{\vdots} \\
    \mathsmaller{r(k)}     & \mathsmaller{r(k+1)} & \mathsmaller{\cdots} & \mathsmaller{r(k-L)} \\
    \mathsmaller{\vdots}   & \mathsmaller{\vdots} & \mathsmaller{\cdots} & \mathsmaller{\vdots} \\
    \mathsmaller{r(N)} & \mathsmaller{r(N-1)} & \mathsmaller{\cdots}  & \mathsmaller{r(N-L)}
%   a & b
\end{pmatrix}
\cdot
\begin{pmatrix}
   \mathsmaller{w(0)}\\
   \mathsmaller{w(1)}\\
   \mathsmaller{\vdots}\\
   \mathsmaller{w(L)}
\end{pmatrix}
=
\begin{pmatrix}
   \mathsmaller{b(L+1)}\\
   \mathsmaller{b(L+2)}\\
   \mathsmaller{\vdots}\\
   \mathsmaller{b(k)}\\
   \mathsmaller{\vdots}\\
   \mathsmaller{b(N)}
\end{pmatrix}
\end{equation}

Notice that $\bm{R}$ is not square for any value of $L$, so the system might have much more equations than unknowns. Thus, resolving the system for $\bm{w}$ results in resolving it in the approximation of the least squares: $\bm{w} \approx (\bm{R}^{T}\bm{R})^{-1}\bm{R}^{T}~\bm{b}$. Matlab does this automatically for non-square matrices with the command \texttt{w~=~R\textbackslash b}.

Once these coefficients $\bm{w}$ are computed, we can freely use the equation (\ref{eq:bt}) to get the estimated values $\hat{b}_{r}(k)$ of the whole signal (for $k \leq N$, and especially for $k \leq L$, we use the known training sequence directly, instead), but we still need to convert them to a binary sequence. In order to do so, we apply a sign detector on the real-valued estimator of $b$, which gives: $\hat{b}(k) = sign(\hat{b}_{r}(k)) = \{_{-1,~\hat{b}_{r}(k) \leq 0}^{+1,~\hat{b}_{r}(k) > 0}$.

\section{Optimisation of the equaliser's order $L$}
We have now everything needed to reconstruct the key, excepted from $L$. [[[[\#\#\#\#\#\#\#As the accuracy of the recovering of the key will depend on the choice made for the order of the equaliser, we have to \#\#\#\#\#\#\#\#]]]]
%using L so that R is almost square
%advantages and disadvantages of a  big  L
%advantages and disadvantages of a small L 





\begin{thebibliography}{1}

\bibitem{HanOttHjalm}
P. Handel, R. Ottoson, H. Hjalmarsson, \emph{Signal Theory}, KTH, 2012

\end{thebibliography}

\balance

\end{document}
